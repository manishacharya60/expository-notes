\begin{figure}[htbp]
\centering
\begin{tikzpicture}[
    vertex/.style={circle,fill=black,inner sep=3pt},
    contract/.style={circle,draw=blue,dashed,inner sep=8pt},
    scale=0.8
]

% Top row - Petersen to K5
\begin{scope}
    % Petersen graph with contraction groups
    % Outer pentagon
    \foreach \i in {0,1,2,3,4} {
        \node[vertex] (outer\i) at ({90+\i*72}:2) {};
    }
    % Inner pentagram
    \foreach \i in {0,1,2,3,4} {
        \node[vertex] (inner\i) at ({90+\i*72}:0.9) {};
    }
    % Draw edges
    \foreach \i in {0,1,2,3,4} {
        \pgfmathtruncatemacro{\next}{mod(\i+1,5)}
        \draw (outer\i) -- (outer\next);
        \pgfmathtruncatemacro{\nexttwo}{mod(\i+2,5)}
        \draw (inner\i) -- (inner\nexttwo);
        \draw (outer\i) -- (inner\i);
    }
    % Contraction circles
    \foreach \i in {0,1,2,3,4} {
        \node[contract] at ({90+\i*72}:1.45) {};
    }
    \node at (0,-2.8) {Petersen Graph};
\end{scope}

% Arrow
\draw[-Stealth, very thick] (2.8,0) -- (4.2,0);

% K5
\begin{scope}[xshift=7cm]
    \foreach \i in {0,1,2,3,4} {
        \node[vertex] (k5\i) at ({90+\i*72}:1.5) {};
    }
    \foreach \i in {0,1,2,3,4} {
        \foreach \j in {0,1,2,3,4} {
            \ifnum\i<\j
                \draw (k5\i) -- (k5\j);
            \fi
        }
    }
    \node at (0,-2.8) {$K_5$};
\end{scope}

% Bottom row - Petersen to K_{3,3}
\begin{scope}[yshift=-7cm]
    % Petersen graph (no contraction circles this time)
    % Outer pentagon
    \foreach \i in {0,1,2,3,4} {
        \node[vertex] (outer2\i) at ({90+\i*72}:2) {};
    }
    % Inner pentagram
    \foreach \i in {0,1,2,3,4} {
        \node[vertex] (inner2\i) at ({90+\i*72}:0.9) {};
    }
    % Draw edges
    \foreach \i in {0,1,2,3,4} {
        \pgfmathtruncatemacro{\next}{mod(\i+1,5)}
        \draw (outer2\i) -- (outer2\next);
        \pgfmathtruncatemacro{\nexttwo}{mod(\i+2,5)}
        \draw (inner2\i) -- (inner2\nexttwo);
        \draw (outer2\i) -- (inner2\i);
    }
    \node at (0,-2.8) {Petersen Graph};
\end{scope}

% Arrow
\draw[-Stealth, very thick] (2.8,-7) -- (4.2,-7);

% K_{3,3}
\begin{scope}[xshift=6cm,yshift=-7cm]
    % Left partition
    \node[vertex] (a1) at (0,1.5) {};
    \node[vertex] (a2) at (0,0) {};
    \node[vertex] (a3) at (0,-1.5) {};
    % Right partition
    \node[vertex] (b1) at (2.5,1.5) {};
    \node[vertex] (b2) at (2.5,0) {};
    \node[vertex] (b3) at (2.5,-1.5) {};
    % Draw all edges
    \foreach \i in {1,2,3} {
        \foreach \j in {1,2,3} {
            \draw (a\i) -- (b\j);
        }
    }
    \node at (1.25,-2.8) {$K_{3,3}$};
\end{scope}

\end{tikzpicture}
\caption{$K_5$ and $K_{3,3}$ are both minors of the Petersen graph}
\label{fig:petersen-minor}
\end{figure}